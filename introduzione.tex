\pagestyle{fancy}
\renewcommand{\chaptermark}[1]{\markboth{#1}{}}
\renewcommand{\sectionmark}[1]{\markright{#1}{}}
\fancyhf{}
\rhead{\rightmark}
\cfoot{\thepage}

\phantomsection
\addcontentsline{toc}{chapter}{Introduzione}
\chapter*{Introduzione\markboth{}{Introduzione}}
Lo sviluppo di applicazioni in ambito mobile ha subito una crescita esponenziale da quando, nel 2008, Apple ha permesso di accedere agli strumenti di sviluppo interni (nello specifico il software Xcode ed i relativi SDK) e ha inserito nella versione 2.0 del software iOS l’applicazione App Store, permettendo di fatto a chiunque di rendere la propria idea una realtà, supportata da un’infrastruttura di servizi ben collaudata, concentrando i propri sforzi solamente sullo sviluppo del software.\\\\In concomitanza anche Google,tramite l’Android Market, ha permesso la pubblicazione di applicazioni sul proprio store, creando così una “corsa all’oro” che solamente ora, 8 anni dopo, sta subendo un sensibile calo dopo anni di crescita significativa e costante.\\\\Precedentemente a queste due piattaforme lo sviluppo in ambito mobile era frenato da fattori importanti quali la mancanza di software di sviluppo stabili e continuamente aggiornati, i sistemi operativi non  sufficientemente avanzati e la scarsa potenza dei dispositivi dell’epoca.\\\\Ciò che ha portato a questa tesi è stato un interesse molto forte verso questo mondo con nemmeno una decade sulle spalle (se si parla di smartphone) e ancora pieno di sviluppi, la maggior parte di questi solo annunciati e ancora in piena fase di progettazione (si pensi per esempio alla realtà aumentata, alle funzionalità appena annunciate sui nuovi dispositivi come le doppie fotocamere, i microfoni HAAC o i sensori per il touchscreen in 3 dimensioni).\\\\In questa tesi abbiamo focalizzato l’attenzione sull’ecosistema creato da Apple per i propri dispositivi iOS ed in particolare sui linguaggi utilizzati per lo sviluppo, Objective-C e Swift. E' stata inoltre creata un'applicazione dimostrativa scritta in Swift, per valutare le potenzialità del nuovo linguaggio in ambito reale.

