\chapter{Objective-C}
Creato agli inizi degli anni '80, Objective-C ha guadagnato la sua popolarità come il linguaggio principale per il sistema operativo NEXTSTEP.\\
Quando nel 1996 Next venne acquisita da Apple il loro sistema operativo divenne la base del nuovo Mac OS, rendendo così questo linguaggio la chiave per lo sviluppo negli anni a venire. Per molti anni Objective-C è rimasto l’unico linguaggio per lo sviluppo su piattaforme Mac ed iOS.\\
La sua implementazione è basata sul linguaggio C con l'aggiunta di caratteristiche di programmazione orientata agli oggetti ed un runtime dinamico.
\section{Caratteristiche del linguaggio}
Questo capitolo descrive le funzionalità che il linguaggio aggiunge allo standard C. 
\subsection{Sintassi}
\subsubsection{Implementazione di una classe, i file .h e .m}
L'implementazione di una classe avviene attraverso due file distinti:
\begin{itemize}
\item Un file con estensione .h, contenente la dichiarazione dell'interfaccia di classe, i metodi e le \textit{properties} pubbliche.
\item Un file con estensione .m, contenente l'implementazione della classe, la definizione di metodi e \textit{properties} private; in questo file inoltre sono definiti i metodi e le variabili d'istanza private.
\end{itemize}
\newpage
Esempio di file header (.h) di una classe:
\lstset{language=[Objective]C, breakindent=40pt, breaklines}
\begin{lstlisting}
#import <Foundation/Foundation.h>

@interface Persona: NSObject ()

//costruttore personalizzato con parametri 

- (id)initConNome:(NSString *)unNome
      cognome:(NSString *)unCognome;

// dichiarazione delle properties private 

@private
	NSString *_nome;
	NSString *_cognome;
	int _eta; 
}

// dichiarazione dei metodi getter e setter

- (NSString *)nome;
- (void)setNome:(NSString *)nome;
- (NSString *)cognome;
- (void)setCognome:(NSString *)cognome;
- (int)eta;
- (void)setEta:(int)eta;

@end
\end{lstlisting}
\bigskip
\bigskip
\bigskip
Esempio di file di implementazione (.m) di una classe:
\lstset{language=[Objective]C, breakindent=40pt, breaklines}
\begin{lstlisting}
#import "Persona.h" 

@implementation Persona

//implementazione della classe dichiarata in Persona.h

- (id)initConNome:(NSString *)unNome
      cognome:(NSString *)unCognome {
    
    if( self = [super init] )
    {
        _nome = unNome;
        _cognome = unCognome;
    }
    
    return self;
}
 
-(NSString *) nome {
	return _nome
}

-(NSString *) cognome {
	return _cognome
}

-(int) eta {
	return _eta
}

-(void)setNome:(NSString *)nome {
	_nome = nome;
}

-(void)setCognome:(NSString *)cognome {
	_cognome = cognome;
}

-(void)setEta:(int)eta {
	_eta = eta;
}
@end
\end{lstlisting}
\bigskip
\bigskip
\bigskip
Attraverso la parola chiave @property è anche possibile dare direttiva al compilatore di definire i setter e i getter:
\lstset{language=[Objective]C, breakindent=40pt, breaklines}
\begin{lstlisting}
//come dichiarati in Persona.h
- (NSString *)nome;
- (void)setNome:(NSString *)nome;

//con property diventa
@property NSString * nome;
\end{lstlisting}
\newpage
Inoltre, attraverso la parola chiave \textit{@synthesize}, è possibile indicare al compilatore di implementare i \textit{setter} e i \textit{getter} di una \textit{property} che sia stata definita precedentemente:
\lstset{language=[Objective]C, breakindent=40pt, breaklines}
\begin{lstlisting}
//come dichiarati in Persona.m

-(NSString *) nome {
	return _nome
}

-(void)setNome:(NSString *)nome {
	_nome = nome;
}
//con synthesize diventa:

@synthesize nome = _nome; 
\end{lstlisting}
Le \textit{properties} permettono di accedere ai \textit{getter} e ai \textit{setter} utilizzando la notazione puntata, per una maggiore leggibilità del codice:
\lstset{language=[Objective]C, breakindent=40pt, breaklines}
\begin{lstlisting}
//senza properties: 

int eta = [persona eta]; 
[persona setEta:22]; 

//con properties: 

int eta = persona.eta;
persona.eta = 22;
\end{lstlisting}
\newpage
\subsubsection{Dichiarazione e definizione dei metodi}
Come visto per la definizione dei \textit{setter}, una dichiarazione di funzione (metodo) ha la seguente sintassi: 
\lstset{language=[Objective]C, breakindent=40pt, breaklines}
\begin{lstlisting}
//Nell'ordine: (tipoDiRitorno) nomeMetodo:(tipoArg1)nomeArg1;

- (NSInteger)calcolaEta:(NSDate)dataDiNascita;
\end{lstlisting}
Definizione del metodo appena dichiarato: 
\lstset{language=[Objective]C, breakindent=40pt, breaklines}
\begin{lstlisting}
- (NSInteger)calcolaEta:(NSDate)dataDiNascita {

	NSDate* oggi = [NSDate date];
	NSDateComponents* componentiCalendario = [[NSCalendar currentCalendar] 
                                   			components:NSCalendarUnitYear 
                                   			fromDate:dataDiNascita
                                   			toDate:oggi
                                   			options:0];
	
	NSInteger eta = [componentiCalendario year];
	
	return eta;
}
\end{lstlisting}
\subsubsection{Invio dei messaggi}
In Objective-C, a differenza della maggior parte degli altri linguaggi, un metodo non viene chiamato direttamente, ma si invia un messaggio all'oggetto stesso: 
\lstset{language=[Objective]C, breakindent=40pt, breaklines}
\begin{lstlisting}
[persona calcolaEta:dataDiNascita];
\end{lstlisting}
Questo è reso possibile dal \textit{runtime} del linguaggio che, tramite una lista, mantiene traccia di tutti i metodi e funzioni che conosce. Ogni elemento della lista è composto da due campi: il nome del metodo (conosciuto come il \textit{selector} dello stesso) e la sua locazione in memoria.
\\Durante la fase di compilazione del codice sorgente, quando vi è la chiamata di un metodo, il compilatore sostituisce il frammento di codice nel seguente modo:
\lstset{language=[Objective]C, breakindent=40pt, breaklines}
\begin{lstlisting}
[persona calcolaEta:dataDiNascita];
\end{lstlisting}
in \lstset{language=[Objective]C, breakindent=40pt, breaklines}
\begin{lstlisting}
objc_msgSend(persona, @selector(calcolaEta:),dataDiNascita);
\end{lstlisting}
\bigskip
\bigskip
La funzione \textit{objc\_msgSend()} opera tramite una ricerca dinamica: conoscendo il nome del metodo da ricercare scorre la lista per verificare la sua effettiva presenza e, in caso affermativo, lo esegue.\\ 
Questo comportamento ha caratteristiche particolari: si potrebbe far puntare un certo selettore A, che prima puntava al codice per A, ad un codice per un certo B.\\ 
Lo svantaggio è che il tempo di esecuzione è leggermente maggiore di una chiamata diretta alla parte di codice; si tratta comunque di nanosecondi di differenza.\\
\\Inoltre l'ambiente di esecuzione, prima di inviare il messaggio, richiede all’oggetto se il metodo è riconosciuto dallo stesso. Questo significa che l’oggetto può decidere se accettare il messaggio (è anche per questo che è possibile inviare messaggi a \textit{nil}), inoltrarlo ad un oggetto differente, decidere di eseguire codice differente per un metodo specifico.\\Più in dettaglio, quando inviamo un messaggio ad un oggetto, non abbiamo garanzie sul fatto che:\\ \\-Il metodo che andiamo a chiamare non sia necessariamente quello che verrà effettivamente chiamato\\\\-L’oggetto che riceverà il messaggio non sarà necessariamente quello che vogliamo che risponda.
\subsubsection{Protocolli}
Un protocollo è una lista di metodi che una classe può implementare; i metodi possono essere resi opzionali grazie all'utilizzo della parola chiave \textit{@optional} prima della loro dichiarazione: 
\lstset{language=[Objective]C, breakindent=40pt, breaklines}
\begin{lstlisting}
@protocol AlarmProtocol <NSObject>

- (void)riproduciSuono;
- (void)fermaRiproduzione;

@optional
- (void)controllaStato;

@end
\end{lstlisting}
\newpage
Una classe deve conformarsi al protocollo prima di implementare i metodi in esso contenuti:\\
File .h: 
\lstset{language=[Objective]C, breakindent=40pt, breaklines}
\begin{lstlisting}
#import "AlarmProtocol.h"
@interface AlarmClass : NSObject  <AlarmProtocol>
\end{lstlisting}
File .m:
\lstset{language=[Objective]C, breakindent=40pt, breaklines}
\begin{lstlisting}
#import "AlarmClass.h"
@implemenation AlarmClass
  
   //implementazione opzionale 
    
    -(void)controllaStato{
		//
    }

    //implementazione obbligatoria  
   
    -(void)riproduciSuono{
    	    // 
    }
    
    -(void)fermaRiproduzione{
   	 	//
    }
@end
\end{lstlisting}
\subsubsection{Tipizzazione dinamica}
Objective-C permette di inviare ad un oggetto un messaggio non specificato nella sua interfaccia; quest'ultimo può a sua volta inviarlo ad un altro oggetto, che può rispondere correttamente od inviarlo a sua volta; questo concetto è chiamato \textit{forwarding}, ovvero inoltro o delega del messaggio.\\Per verificare se un dato oggetto può rispondere viene utilizzato il metodo di NSObject \textit{- (BOOL)respondsToSelector:(SEL)aSelector;}, che indica con un valore booleano se il ricevente implementa o eredita un metodo che può rispondere al messaggio.\\Alternativamente, è possibile utilizzare un gestore degli errori nel caso il messaggio non possa essere inoltrato. Se l'oggetto non inoltra, non gestisce l'errore o non risponde viene generato un errore di \textit{runtime}. 
\subsubsection{Categories}
Una categoria raccoglie implementazioni di metodi in file separati. La particolarità è che questi metodi sono aggiunti ad una data classe a \textit{runtime}; ciò permette di aggiungere metodi ad una classe esistente senza doverla ricompilare o senza avere a disposizione il suo codice sorgente.\\I metodi inseriti nelle categorie diventano indistinguibili da quelli originari della classe quando il programma è in esecuzione.\\ Una categoria ha pieno accesso a tutte le variabili d'istanza della classe, anche quelle private. 
\lstset{language=[Objective]C, breakindent=40pt, breaklines}
\begin{lstlisting}
//interfaccia della category che estende la classe NSString
//aggiungendo una nuova funzione chiamata reverseString 

@interface NSString (reverse)
-(NSString *) reverseString;
@end
\end{lstlisting}
\subsubsection{La direttiva import}
Permette di inserire un file prima dell'inizio della compilazione, ed evita di includere il file qualora sia già stato incluso in precedenza.\\Viene comunemente utilizzata per includere i file \textit{header} delle classi nei file .m contenenti le implementazioni delle interfacce in esso contenute.
\subsection{Compilatore}
L'IDE di Apple, Xcode, utilizza clang. Quest'ultimo è un compilatore \textit{front-end} per C, C++, Objective-C, Objective-C++, OpenMP, OpenCL e CUDA, basato sul compilatore LLVM (Low Level Virtual Machine).\\E' stato creato in quanto si era reso necessario un compilatore che avesse migliori strumenti di diagnostica, migliore integrazione con gli strumenti di sviluppo, una licenza che permettesse di utilizzarlo in programmi commerciali ed infine, che fosse semplice da sviluppare e mantenere. GCC (GNU Compiler Collection), che inizialmente doveva essere utilizzato, era considerato lento e di difficile utilizzo.\\\\Gli obiettivi principali di clang sono quattro:
\begin{itemize} 
\item Avere un \textit{parser} unificato per i linguaggi basati su C, fornendo al contempo buoni strumenti di diagnostica.
\item Avere un'architettura basata su una libreria di API scritte in C++, con la possibilità di estensioni.
\item Essere polivalente, includendo strumenti di indicizzazione, analisi statica del codice e \textit{refactoring}
\item Avere ottime \textit{performance}, garantendo un utilizzo parsimonioso della memoria, con tempi di compilazione rapidi.
\end{itemize}
Attualmente viene considerato un compilatore solido e stabile per codice C, Objective-C, C++ e Objective-C++, se utilizzato per le piattaforme X86-32, X86-64, ed ARM.  
\subsection{Utilizzo in iOS e frameworks Cocoa}
Il linguaggio principale di tutti i frameworks (quali CoreOS, Foundation, CocoaTouch) di iOS e MacOS è ancora Objective-C come dal principio, in quanto Swift non ha ancora raggiunto un livello di maturità tale per essere impiegato per questi scopi (la versione 3 del linguaggio non ha ancora ABI stabili e la sintassi è ancora in corso di modifiche). L'intenzione di Apple è di mantenere e migliorare i due linguaggi parallelamente per ancora molto tempo.