\chapter{Il futuro della piattaforma iOS}
Con più di un miliardo di dispositivi attivi, iOS è sicuramente una delle piattaforme mobile di riferimento.\\
Oltre ai device mobili con i quali questo sistema operativo è nato, sono stati nel corso del tempo introdotti ulteriori dispositivi che utilizzano la piattaforma od una sua versione modificata, tra i quali Apple Watch (WatchOS), Apple TV (TvOS).\\\\Per quanto riguarda i linguaggi di programmazione, attualmente Objective-C è ancora il linguaggio più utilizzato e lo sarà ancora per gli anni a venire, in quanto Swift non ha ancora introdotto ABI (Application Binary Interface) stabili, che ne pregiudicano l'utilizzo nei \textit{frameworks} e conseguentemente non viene ancora utilizzato da Apple per scrivere parti del sistema operativo. Al momento le uniche applicazioni interne scritte in questo linguaggio sono Musica su iOS ed la barra delle applicazioni di sistema (Dock) su MacOS.\\\\
Sta inoltre avvenendo un altro cambiamento importante a livello di piattaforma: la transizione ai 64-bit.\\Dal 2013 i dispositivi Apple hanno integrato processori ARMv8 capaci di supportare questa funzionalità, ed inoltre dalle ultime versioni di iOS viene mostrato un avviso all'utente ad indicare che l'applicazione aperta, in quanto compilata solamente per gli ambienti a 32-bit, potrebbe non funzionare in futuro.\\\\
In quanto linguaggio \textit{open source}, è possibile capire quali saranno le caratteristiche della nuova versione di Swift, in particolare la quarta iterazione che sarà rilasciata nel tardo 2017.\newpage
Lo scopo principale della nuova \textit{release} sarà di garantire compatibilità di sorgenti con la versione attuale, e di portare stabilità alle interfacce della libreria standard; il rilascio avverrà in due fasi distinte.\\\\
Nella prima fase l'interesse principale sarà focalizzato sulla suddetta stabilità delle interfacce; verranno inoltre inclusi miglioramenti ai tipi \textit{generics}, alle stringhe, e sarà introdotto un modello di gestione della memoria opzionale ispirato ai linguaggi Cyclone e Rust che permetterà, in particolare agli sviluppatori di sistema e di applicazioni ad alte prestazioni, un controllo predicibile e deterministico delle performance.\\\\
La seconda fase sarà focalizzata sul fornire supporto al codice sorgente della versione precedente; se si renderanno necessari cambiamenti che andranno a minare la retrocompatibilità, verrà utilizzato un flag "Swift 4" per il compilatore.\\Verranno inoltre introdotte migliorie alla libreria standard, in particolare l'area di interesse riguarda le strutture dati e gli algoritmi utilizzati. Sono inoltre menzionati miglioramenti al supporto dei \textit{frameworks} di Foundation API, per migliorare la compatibilità di Cocoa con Swift; al momento della scrittura di questa tesi non sono stati forniti ulteriori dettagli.
 
