\pagestyle{fancy}
\renewcommand{\chaptermark}[1]{\markboth{#1}{}}
\renewcommand{\sectionmark}[1]{\markright{#1}{}}
\fancyhf{}
\rhead{\rightmark}
\cfoot{\thepage}

\phantomsection
\addcontentsline{toc}{chapter}{Conclusioni}
\chapter*{Conclusioni\markboth{}{Conclusioni}}
Swift è un linguaggio che sta guadagnando sempre più trazione, ma affinché cominci a prendere il sopravvento è necessario attendere la stabilità delle ABI, che porteranno gli sviluppatori di frameworks (interni ed esterni) a considerarlo per lo sviluppo in quanto linguaggio stabile.\\\\
Objective-C rimane un linguaggio molto conosciuto e performante, con caratteristiche non ancora importate da Swift. La sintassi particolare e l’ereditarietà del C lo rendono poco appetibile ai neofiti della piattaforma e della programmazione in generale.\\\\
Il consiglio per chi si approccia al mondo iOS è comunque quello di utilizzare Swift (dalla versione 3 e successivi) per la sua sintassi espressiva e concisa, valutando attentamente la presenza di frameworks aggiornati che supportino il linguaggio. Si renderà probabilmente necessario utilizzare files in Objective-C, ma l’interoperabilità tra i due porta comunque un vantaggio se si utilizza Swift.
\\Esperienza nello sviluppo: parlare del passaggio di versioni swift, cocoapods che non vengono aggiornati subito, problemi vari con xcode