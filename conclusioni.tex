\pagestyle{fancy}
\renewcommand{\chaptermark}[1]{\markboth{#1}{}}
\renewcommand{\sectionmark}[1]{\markright{#1}{}}
\fancyhf{}
\rhead{\rightmark}
\cfoot{\thepage}

\phantomsection
\addcontentsline{toc}{chapter}{Conclusioni}
\chapter*{Conclusioni\markboth{}{Conclusioni}}
Swift sta guadagnando sempre più trazione, ma affinché cominci la vera fase di transizione sarà necessario attendere la stabilità delle ABI (Application Binary Interface), che porteranno gli sviluppatori di librerie, interni ed esterni, a considerarlo per lo sviluppo in quanto linguaggio con una base stabile.\\\\
Objective-C d'altra parte rimane un linguaggio molto conosciuto, robusto e performante, con caratteristiche non ancora importate da Swift. La sintassi particolare e l'ereditarietà del C lo rendono però poco appetibile ai neofiti della piattaforma e della programmazione in generale.\\\\
Lo scopo dell'applicazione creata durante il tirocinio è stato di valutare quali differenze emergono sviluppando nei due linguaggi: da una parte uno pienamente maturo ed utilizzato a tutti i livelli, dall'altra uno in piena evoluzione e soggetto a modifiche repentine.\\\\
Il consiglio per chi si approccia allo sviluppo in ambiente iOS è comunque di considerare l'utilizzo di Swift (dalla versione 3 e successivi) per la sua sintassi espressiva e concisa, valutando attentamente la presenza di frameworks aggiornati; nonostante sia appena alla terza versione è già possibile creare applicazioni stabili ed efficienti, ma soprattutto creare un codice più semplice, scorrevole e leggibile, utilizzando una sintassi al passo con i tempi.\\\\
Inoltre, l'interoperabilità tra i due linguaggi permetterà un passaggio incrementale per chi possiede già esperienza nello sviluppo di applicazioni con Objective-C.