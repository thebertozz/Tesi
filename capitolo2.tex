\chapter{Swift}

\section{Cenni storici}
Lo sviluppo di Swift è iniziato nel 2010 da Chris Lattner, aiutato in seguito da molti altri programmatori. Swift ha preso idee "da Objective-C, Rust, Haskell, Ruby, Python, C\#, CLU, e molti altri".Il 2 giugno 2014 l'app per il WWDC è divenuta la prima app distribuita al pubblico scritta in Swift.\\Il 3 dicembre 2015 viene lanciato il sito swift.org ed il codice sorgente del linguaggio è pubblicato con licenza Apache 2.0 sul repository GitHub dell'azienda.\\Apple resta lo sviluppatore principale e ne rende disponibile anche una versione del compilatore per Linux (Creato appositamente per Ubuntu).\\Il 13 settembre 2016, durante la WWDC 2016, Apple ha presentato la terza versione del suo linguaggio di programmazione insieme ad un'applicazione per iPad, Swift Playgrounds, che permette, tramite una grafica semplice e intuitiva, di imparare a programmare con Swift, soprattutto orientato ai più giovani.
\section{Caratteristiche}
\subsection{Sintassi}
\subsubsection{Implementazione di una classe}
In Swift, al contrario di molti altri linguaggi, non ci sono 2 file distinti per l'interfaccia e l'implementazione, ma uno solo con l'estensione .swift\\
Esempio di creazione di una nuova classe:\\
\begin{lstlisting}
class Persona { 

//dichiarazione delle properties

	var nome: String? 
	var cognome: String? 
	var eta: Int?

//costruttore personalizzato con parametri 

	init(nome: String, cognome: String, eta: Int) {
		self.nome = nome
		self.cognome = cognome
		self.eta = eta
	}

//dichiarazione dei metodi setter e getter 

	func getNome() -> String {

		return self.nome
	}

	func setNome(nome: String) {

		self.nome = nome
	}

	func getCognome() -> String {

		return self.cognome
	}

	func setCognome(cognome: String) {
	
		self.cognome = cognome
	}

	func getEta() -> Int {
	
		return eta
	}

	func setEta(eta: Int) {

		self.eta = eta
	}
}
\end{lstlisting}
\subsubsection{Dichiarazione e definizione dei metodi}
In Swift, una dichiarazione di funzione (metodo) ha la seguente sintassi: 
\begin{lstlisting}
//Nell'ordine: func nomeMetodo(nomeArg1:tipoArg1) -> tipoDiRitorno

func calcolaEta(dataDiNascita: NSDate) -> Int

\end{lstlisting}
Definizione del metodo appena dichiarato: 
\begin{lstlisting}
func calcolaEta(dataDiNascita: NSDate) -> Int

	let oggi = Date()
      
    let componentiCalendario = Calendar.current.dateComponents([.year], from: dataDiNascita, to: oggi)                             			
	
	let eta = componentiCalendario.year!
	
	return eta;
}
\end{lstlisting}
Le funzioni in Swift sono trattate come oggetti, ciò significa che una funzione può ritornare un'altra funzione: 
\begin{lstlisting}
func creaIncrementatore() -> ((Int) -> Int) {
	
	func aggiungiUno(numero: Int) -> Int {
	
		return 1 + numero 
	
	}
	
	return aggiungiUno
}

var incrementatore = creaIncrementatore()
incrementatore(7)
\end{lstlisting}
\subsubsection{Closures}
Le funzioni sono un caso speciale di "closure": il codice in una closure ha accesso a variabili e funzioni che sono disponibili nel suo scope, anche se viene eseguita in uno scope diverso. Una closure viene definita dalla sintassi \{ \}, utilizzando il separatore \"in\" per gli argomenti e il tipo di ritorno dal corpo:\\\\ 
\begin{lstlisting}
var numeri = [2,25,21,89,90]

numeri.map({
	(numero: Int) -> Int in 
	let risultato = 3 * numero 
	return risultato 
})
\end{lstlisting}
Ci sono vari modi per scrivere le closure: quando il tipo è già conosciuto, come per esempio in una callback per un delegate, si possono omettere i tipi dei parametri, il tipo di ritorno o entrambi nel caso in cui ci sia un singolo statement, in quanto la closure ritorna implicitamente il valore di ritorno:
\begin{lstlisting}
let numeriInMap = numeri.map({ numero in 3 * numero })
\end{lstlisting}
Ci si può riferire ai parametri per numero invece che per nome, approccio utile specialmente in closure che richiedono poco codice; una closure passata come ultimo argomento di una funzione può essere scritta immediatamente dopo le parentesi, e se quest'ultima è l'unico argomento della funzione stessa si possono omettere le parentesi tonde: 
\begin{lstlisting}
//ordino i numeri in modo crescente

let numeriOrdinati = numeri.sorted { $0 > $1 }
\end{lstlisting}
\subsubsection{Enumerazioni}
La sintassi enum è utilizzata per dichiarare le enumerazioni in Swift. La particolarità rispetto ad Objective-C è che quest'ultime possono contenere metodi: 
\begin{lstlisting}
enum TipologieDiCase {

	case condominio, villa, indipendente, attico
	
	func descrizione() -> String {
	
		switch self {
	
		case .condominio:
			return "Condominio"
		case .villa :
			return "Villa"
		case .indipendente:
			return "Casa indipendente"
		case .attico
			return "Attico"
		default: 
			return String(self.rawValue)
		
		}
	}
}

let villa = TipologieDiCase.villa
let descrizioneVilla = villa.descrizione()
\end{lstlisting}
\subsubsection{Protocolli ed estensioni}
Un protocollo definisce un'interfaccia di metodi, variabili ed altri eventuali requisiti che definiscono una particolare funzionalità. Quest'ultimo può essere quindi adottato da una classe, struct o enum che ne forniranno l'implementazione:
\begin{lstlisting}
protocol ProtocolloDiNavigazioni {
	
	func navigaAlleImpostazioni(sender: CollectionView) 
	func navigaAlDettaglioEvento(sender: UICollectionView)
}

class MenuPrincipale: UICollectionViewController, UICollectionViewDelegateFlowLayout: ProtocolloDiNavigazioni {

	override func viewDidLoad() {
        super.viewDidLoad()
		collectionView?.delegate = self
		collectionView?.dataSource = self
	}
	
	...
	
	//MARK: CollectionViewDelegate 
	
	override func collectionView(collectionView: UICollectionView, didSelectItemAtIndexPath indexPath: NSIndexPath) {
		
		switch indexPath {
			
			case 0: 
				navigaAlleImpostazioni(self.collectionView)
			case 1:
				navigaAlDettaglioEvento(self.collectionView)
		}
	}	
		
	//MARK: Implementazione del protocollo 
		
	func navigaAlleImpostazioni(sender: UICollectionView) {
        
        appDelegate.gotoSettingsVC()
    }
    
    func navigaAlDettaglioEvento(sender: UICollectionView) {
        
        appDelegate.gotoEventDetailVC()
    }
}
\end{lstlisting}
Le estensioni sono invece un modo per aggiungere funzionalità ad un tipo eisstente, come nuovi metodi e computed properties:
\begin{lstlisting}
//Estensione che aggiunge un effetto di blur ad una imageView

extension UIImageView
{
    func aggiungiBlur()
    {
        let blurEffect = UIBlurEffect(style: UIBlurEffectStyle.Light)
        let blurEffectView = UIVisualEffectView(effect: blurEffect)
        blurEffectView.frame = self.bounds
        self.addSubview(blurEffectView)
    }
}

let containerImmagine = UIImageView()
containerImmagine.image = UIImage(named: "beer2beerlogo.jpg")
containerImmagine.aggiungiBlur()
\end{lstlisting}
\subsubsection{Gestione degli errori}
Gli errori vengono rappresentati utilizzando qualsiasi tipo che si conformi al protocollo Error; la parola chiave throws viene utilizzata per indicare che una funzione può ritornare un errore, utilizzando la parola chiave throw. Se si lancia un errore dall'interno di una funzione, quest'ultima ritorna immediatamente e l'errore viene gestito dalla funzione chiamante:
\begin{lstlisting}
enum ErroriStampante: Error {

	case cartaEsaurita
	case inchiostroEsaurito
	case cassettoChiuso
	
}

func invia(lavoro: Int, allaStampante nomeStampante: Stringa) throws -> String {
	if nomeStampante = "Rusty old printer" {
		
		throw ErroriStampante.cassettoChiuso
	}

	return "Lavoro inviato alla stampante"
}

	do {
	
	let rispostaStampante = try invia(lavoro: 2303, allaStampante: "Sala meeting")
	
	print(rispostaStampante)
	
	} catch {
		
		print(error)
	}
	
\end{lstlisting}
Si possono inoltre utilizzare più blocchi catch per gestire errori specifici:
\begin{lstlisting}
	do {
	
		let rispostaStampante = try invia(lavoro: 2303, allaStampante: "Sala 	meeting")
	
		print(rispostaStampante)
	
		} catch ErroriStampante.cassettoChiuso {
		
			print("Aprire il cassetto")
		}
\end{lstlisting}
Un altro modo per gestire gli errori è quello di utilizzare la parola chiave try? per convertire il risultato in un tipo optional: se la funzione lancia un errore, questo specifico errore è ignorato e la funzione ritorna nil; alternativamente il risultato è un optional contenente il valore ritornato dalla funzione:
\begin{lstlisting}
let foglioStampato = try? invia(lavoro: 1984, allaStampante: "Sala meeting")
let erroreDiStampa = try? invia(lavoro: 1948, allaStampante: "Rusty old printer")
\end{lstlisting}
\subsection{Gestione della memoria}
TODO
\subsection{Compilatore}
TODO
\subsection{Utilizzo in iOS e frameworks Cocoa}
TODO